\documentclass[a4paper]{article}
\setlength{\headheight}{1.1\baselineskip}
\usepackage[english]{babel}
\usepackage[utf8x]{inputenc}
% package for including graphics with figure-environment
\usepackage{graphicx}
\usepackage{hyperref}
% colors for hyperlinks
% colored borders (false) colored text (true)
\hypersetup{colorlinks=true,citecolor=black,filecolor=black,linkcolor=black,urlcolor=black}

% package for bibliography
\usepackage[authoryear,round]{natbib}
% package for expectation signs
\usepackage{amsmath,amssymb,mathtools,bm,etoolbox}
%\documentclass[a4paper,11pt]{report} 
\usepackage{breqn}
\usepackage{amsmath}
\usepackage{enumitem} 
\usepackage{amsmath, amsthm, amssymb}
\usepackage{amsmath}
\newcommand{\abs}[1]{ \left\lvert#1\right\rvert} 
\newcommand{\norm}[1]{\left\lVert#1\right\rVert}
% package for header
\usepackage[automark]{scrpage2}
\pagestyle{scrheadings}
\ihead[]{Isa Marques, Xi Sun, Xueying Liu}
\ohead[]{\today}
\cfoot[]{\pagemark} 
\setheadsepline[122mm]{0.3mm}

\begin{document}
	\title{
	%\begin{figure}[!ht]
	%	\flushleft
	%		\includegraphics[width=0.7\textwidth]{logo.eps}
	%\end{figure}
	\vspace{1cm}
	\Huge \textbf{ Semiparametric Single Index Models }\\ \Large Ichimura and Klein and Spady's methods \\
	}
	
	\vspace{1cm}
	
	% if you are the only author, you might use the following
	% \author{Name of student}	
	
	% Insert here your name and correct mail address
	\author{\Large \href{mailto:first.student@smail.fh-koeln.de}{Isa Marques}\and \Large \href{mailto:second.student@smail.fh-koeln.de}{Xi Sun} \and \Large \href{mailto:second.student@smail.fh-koeln.de}{Xueying Liu}
	\vspace{1cm}}
	
	% name of the course and module
	\date{
	\large University of Bonn \\ Project Module in Econometrics and Statistics\\ 
	\vspace{0.8cm}
	\large Prof. Dr. Kneip \\
	\large Prof. Dr. Liebl \\
	\vspace{1cm}
	\today
	}

	\maketitle
	\setlength{\parindent}{0pt}

\vspace{2cm}
\begin{abstract}


\end{abstract}
	\newpage
	\tableofcontents
	\newpage
	
\section{Introduction} % (fold)
\label{sec:introduction}
TO BE WRITTEN WHEN THE APPLIED PART IS DONE.
SAY WHO DID WHAT.

Semiparametric single index models are widely applied in economic research. Applications range from finance to labor economics.

% section introduction (end)
The main focus of this paper is Ichimura's semiparametric model. In section 4 the identification conditions necessary for uniquely determining $\beta_0$ and $g(\cdot)$ in semiparametric models are presented.  Furthermore, Ichimura's solution will be analyzed in detail in section 5. The latter section briefly explains the weight function and bandwidth selection. As a comparison to Ichimura's method, Klein and Spady's (1993) semiparametric binary choice model will be explained concisely in section 6. Finally, in section 6 the two models are compared in both a theoretical and applied perspective.



\section{Context} % (fold)
\label{sec:context}
In this section, we elaborate on the model's main features and contributions to the class of semiparametric models. Close at hand, a comparison with classical parametric models and nonparametrics models is provided. In essence, the risk of mispecification is reduced relative to the overly restrictive but interpretable parametric models. Additionally, it avoids inconveniences of fully nonparametric models such as the curse of dimensionality, difficulty of interpretation, and the lack of extrapolation capability. However, this does not come lightly, as computation for semiparameteric single index models is often difficult.

\vspace{2mm} 
The model
 
\begin{eqnarray}
Y_i = g(X_i'\beta) + \varepsilon_i,
\end{eqnarray}

where

\begin{enumerate}[label=(\roman*)]
		\item $\{x_i,y_i\}$ for i = 1, ..., n is an i.i.d. sample;
		\item $Y_{i}$ is the dependent variable, $X_i \in \mathbb{R}^{q}$ is a vector of explanatory variables, $\beta$ is the q $\times$ 1 vector of unknown parameters; 
	\item $X_i'\beta$ is a single index because it is a scalar;
	\item $ E(\varepsilon_i|x_i) = 0 $;
	\item $g: \mathbb{R} \rightarrow \mathbb{R} $ is not known; 
\end{enumerate}

is a single index model.
\vspace{2mm}

Three points become relevant when explaining why the present model is of  semiparametric nature. First, unlike fully nonparametric models, the functional form of the linear index is stated. However, as opposed to classical parametric models, the conditional probability of $\varepsilon$ conditioned on X is not specified except $ E(\varepsilon|X) = 0 $. Alongside, $g(\cdot)$ is left fully unspecified.

For illustrative purposes we now analyze binary choice models in the setting proposed by Li and Racine (2007). Note that Y can be either discrete or continuous. The relationship between a binary dependent variable (Y) and covariates (X) can be modelled as follows:

\[
    Y_i = 
    \begin{cases}
      1, & \text{if}\ Y_i^* \stackrel{def}{=} \alpha + X_i'\beta + u_i > 0 \\
      0, & \text{if}\ Y_i^* = \alpha + X_i'\beta + u_i \leq 0
    \end{cases}
\]

where $Y^{*}$ is a latent variable.
Assuming a linear relationship between Y and X, the empirical analysis focuses on the estimation of $\beta$.
Parametric methods to estimate $\beta$ require assumptions on the distribution of the error term $u$. A common assumption in the parametric framework is $ u \sim N(0, 1)$. \footnote{With the identification condition $\sigma = 1$, $\beta$ can be jointly identified (Madalla (1986)) and we can use maximum likelihood.}  Let $F_u(\cdot)$ denote the true CDF of \textit{u}. Then, the conditional expectation of Y has the form

\[ 
\begin{split}
E(Y|x) & = \sum_{y=0,1} yP(y|x) = 1 \times P(Y=1|x) + 0 \times P(Y=0|x) \\
 & = P(Y=1|x) = P(\alpha + x'\beta + u_i > 0) \\
 & = P(u_i > -(\alpha + x'\beta)) = 1 - P(u_i \leq -(\alpha + x'\beta)) \\
 & = 1 - F(-(\alpha + x'\beta)) \equiv m(\alpha + x'\beta),
\end{split}
\]


where $F(\cdot)$ is the cummulative distribution function (CDF) of $u$. Furthermore, $1 - F(-(\alpha + x'\beta) = F(\alpha + x'\beta)$, since $u$ has a symmetric distribution.

Model (2) is then commonly referred to as a Probit model

\[
E(Y|X) = P(Y=1|x) = \Phi(\alpha + x'\beta),
\]

where $\Phi$ is the CDF of a standard normal variable. 

Alternatively, for $u$ following a symmetric logistic distribution a logit model is obtained. Generally speaking, model (1) contains many widely used parametric models that assume $g$ is known up to a finite-dimensional parameter. Thus, different functional forms for $u$ lead to different functional forms for the conditional probability of $Y = 1$. Abridge, consistent parametric estimation of $E(Y|X) = P(Y=1|x)$ requires the \textit{correct} distribution specification of $u$. 
Hence, model (1) with unknown $g(\cdot)$ is a more flexible version of parametric models which still clings to many of its desirable features. 

Distinctively, nonparametric models can be defined as
\[Y = g(X, \varepsilon)\]
with smooth $g$ and given sampling,  or  
\[Y = g(X) + \varepsilon\]
which additionally assumes additivity of the error. Thus, the assumptions of a single index model are weaker than those of a parametric model and stronger than those of a nonparametric model.  \footnote{However, the necessary assumptions for consistent parametric estimation can be relaxed. In particular, the single index model might have weaker assumptions than a fully parameterized model for structural economic models.} 

Nonparametric models typically suffer from the curse of dimensionality. This is defined by Gery Geenens (2011) as being caused by the sparsity of data in high-dimensional spaces, which results in a decrease in fastest achievable rates of convergence of regression function estimators toward their target curve as the dimension of the vector of independent variables increases. 

The single index model avoids the curse of dimensionality by reducing the p-dimensional predictor to a univariate single-index. In fact, the estimator achieves the same convergence rate $n^{-\frac{1}{2}}$ that is optimal for most parametric models. For nonparametric models this rate is only $n^{-\frac{2}{5}}$, if the underlying function is twice continuously differentiable (Cameron and Trivedi(2005)).
Thus, in general, single index models reach greater estimation precision than fully nonparametric estimators with multidimensional X.



Nonetheless, semiparametric models have two important disadvantages. First, they are hard to compute. Take the case of Ichimura's (1993) model, which requires nonlinear iteration procedures. In fact, iterative procedures are quite often required. Second, semiparametric models often have multiple local optima, as they require optimization of objective functions that are not unimodal. These problems seem to be exarcebated for increasing sample sizes or number of explanatory variables (Manski, 1975, 1985; Manski and Thompson, 1986; Cosslett, 1983; Ichimura, 1993; Horowitz, 1992; and Klein and Spady, 1993).


\section{Identification conditions} % (fold)
\label{sec:Identification conditions}

This section provides identification conditions for semiparametric single index models,  summarized in proposition 3.1. Brief intuitive explanations follow each of these conditions. Moreover, it is under these that $\beta_0$ and $ g(\cdot)$ are estimated in Ichimura (1993) and Klein and Spady's (1993) models, analyzed in sections 4 and 5, respectively.


Model (1) implies

\begin{equation}
E(Y|x) = g(x'\beta_0).
\end{equation}

Thus $Y$ depends on $x$ only through the linear combination $x'\beta_0$, and this relationship is characterized by the link function $g(\cdot)$. Before estimating  $\beta_0$ and $ g(\cdot)$



\newtheorem{prop}{Proposition}[section]

\begin{prop}
Identification of $\beta_0$ and $g(\cdot)$ in model (2) requires that
\begin{enumerate}[label=(\roman*)]
\item The vector of independent variables x should not contain a constant and it must contain at least one continuous variable with nonzero coefficient. Furthermore, one component of $\beta_0$ is set to 1. 
\item The support of $x'\beta_0$ is a bounded convex set with at least one interior point. Function $g$ is differentiable and it is not a constant function on the support of $x'\beta_0$;
\item For the discrete components of $x$, changing the values of the discrete variables will not divide the support of $x'\beta_0$ into disjoint subsets.
\end{enumerate}
\end{prop}

Some intuition is now provided for the restrictions imposed on $x$ in $(i)$. First, $x$ cannot suffer from multicollinearity. That is, $Pr(x'\alpha = c) = 1$ where $\alpha$ is a constant and c is a scalar. Furthermore, requiring $x$ to contain at least one continuous variable (with nonzero coefficient) prevents $x$ from having a finite support. This logic also applies to  the scalar variable $ v = x'\beta_0 $, for any vector $\beta_0$. Otherwise, $E(Y|X = x) = g(x'\beta_0)$ would impose only a finite number of restrictions on $g(\cdot)$, leading to an infinite number of different choices for $g(\cdot)$ and $\beta_0$ that satisfy those restrictions.\footnote{Note that even when all $x$ components are discrete we can still identify bounds on the components of $\beta_0$, if $g$ is assumed to be increasing See Horowitz (1998) for  concrete examples.} 

Analogously, identification requires \textit{location normalization} and \textit{scale normalization}. Define the function $g^{*}$ such that $g^{*}(\gamma + v\delta) = g(v)$, for all $v$ in the support of $x'\beta_0$. Then

\begin{equation}
E(Y|X = x) = g(x'\beta_0)
\end{equation}

and

\begin{equation}
E(Y|X = x) = g^*(\gamma + x'\beta_0\delta).
\end{equation}

Models (3) and (4) are observationally equivalent. Thus, $\beta_0$ and $g$ are not identified unless restrictions are imposed to uniquely specify $\gamma$ and $\delta$. \textit{Location normalization} is obtained by restricting $\gamma$. For example, if $x$ does not include a constant. On the other hand, \textit{scale normalization} restricts $\delta$. Here, it is assumed that $\beta_0$ has one of its components set to 1. \footnote{This implies that $X$ must have at least 2 dimensions. Otherwise $\beta_0$ is simply normalized to 1 and a one-dimensional nonparametric model $E(y|x) = g(x)$ with no semiparametric part is obtained instead.}

As to what concerns part (ii), restrictions are also imposed on $g(\cdot)$, even though some of these can potentially be weakened. First, $g(\cdot)$ cannot be a constant function. Otherwise, $\beta_0$ is not identified. Furthermore, what makes the identification of $E(Y|X = x)$ possible is that it remains constant if $x$ changes in such a way that $x'\beta_0$ stays constant. However, $P(x'\beta_0 = c)=0$, for $x_0'\beta$ continuously distributed and for some constant $c$. This renders identification impossible. By adding the assumption that $g(\cdot)$ is differentiable, $g(x'\beta_0)$ is close to $g(c)$ whenever $x'\beta_0$ is close enough to $c$. Then, the set of $x$ for which $x'\beta_0$ is within any specified nonzero distance of $c$ has nonzero probability for $c$ in the interior of the support of $x'\beta_0$. Therefore, we identify $\beta_0$ by the approximate constancy of $x'\beta_0$. In fact, Wei Lin and Kulasekera (2007) show that the weaker assumption that $g(\cdot)$ is continuous is sufficient for identification. However, differentiability is assumed on the remainder of the paper as it will become useful when analyzing Ichimura's (1993) and Klein and Spady's (1993) models.

The need for condition (iii), that is,  the need to prevent $x'\beta_0$  from being divided into disjoint subsets, can be explained with an example inspired on Horowitz (1998). Consider a Single Index Model in which X has a continuous component $X_1$ with support $\big[0,1\big]$, and one discrete component $X_2$ whose support is $\{0,1\}$. Assume $X_1$ and $X_2$ are independent, $g(\cdot)$ is strictly increasing and non periodic and set $\beta_1 = 1$ as a \textit{scale normalization}. 

Consider in particular the case
\[
\begin{split}
E[Y| X = (x_1,0)]& = g(x_1), \text{support\ of } g(\cdot): [0,1];  \\
E[Y| X = (x_1,1)]& = g(x_1+\beta_2), \text{support\ of } g(\cdot): [\beta_2,1+\beta_2].
\end{split}
\]

For $X_2 = 0$ the function $g(\cdot)$ is identified on $\big[0,1\big]$. However, for $\beta_2 > 1$ the support of $ X_1 + \beta_2$ is disjoint from $\big[0,1\big]$ and $\beta_2$ is an intercept in the model for $E(Y|(x_1,1))$. Therefore, it is not possible to identify $\beta_2$. However, for $\beta_2 < 1$ the support of $X_1$ and $X_1 + \beta_2$ overlap. In particular, for $0<\beta_2<1$  and the subset $\{X: X_1 \in [\beta_2, 1] \wedge X_2 = 1\}$, $g(x_1 + \beta_2) = g(v)$ for some $v \in [0,1]$. \footnote{For $\beta_2$ strictly negative, one can consider $\{X: X_1 \in [ - \beta_2, 1] \wedge X_2 = 1\}$.} Then, $g(v)$ can be identified for $v \in [\beta_2, 1]$ by observations of $X_1$ for which $X_2 = 0$. 

To sum up, $\beta_2$ can be identified by solving

\begin{equation}
E[Y| X = (x_1,1)] = g(x_1 + \beta_2),
\end{equation}

for $x_1 \in \big[\beta_2, 1\big]$. \footnote{Note that if g was periodic on $\big[0, 1 - \beta_2 \big]$, (7) would have at least two solutions and $\beta_2$ would not be identified.}


\section{Ichimura's estimation model} % (fold)
\label{sec:Ichimura's estimation model}

In this section Ichimura's (1993) estimation method for semiparametric models is analyzed. This method exhibits $n^{-\frac{1}{2}}$ efficiency and asymptotic normality. A weighting matrix that reaches the semiparametric efficiency bound is investigated. Nonetheless, multiple local minima may result. 

Let $\beta_0$ denote the true value of $\beta$. For known $g$, $\beta_0$ is estimated by minimizing the nonlinear least squares (NLS) problem

\[
S(\beta) = E[Y - g(x'\beta)]^2. \]

Exclusively as a means to understand the latter statement, assume $E[\varepsilon^2|X]=\sigma^2$. Then,
\begin{align*}
E[(Y - g(x'\beta))^2] & = E[\{Y - g(x'\beta_0) + (g(x'\beta_0) - g(x'\beta))\}^2]\\
                   & = E[\varepsilon^2] + 2E[\varepsilon (g(x'\beta_0) - g(x'\beta)) ] + E[(g(x'\beta_0) - g(x'\beta))^2]\\                 & = \sigma^2 + E[E[\varepsilon (g(x'\beta_0) - g(x'\beta))|X] + E[(g(x'\beta_0) - g(x'\beta))^2]  \\
                   &= \sigma^2 + E[(g(x'\beta_0) - g(x'\beta))E[\varepsilon|X]] + E[(g(x'\beta_0) - g(x'\beta))^2] \\
                   &= \sigma^2 + E[(g(x'\beta_0) - g(x'\beta))^2]
\end{align*}

Where the last equality follows from the assumption $E[\varepsilon|X]=0$. Hence, given the scale and location normalization conditions from Proposition 3.1., $E[(Y - g(x'\beta))^2]$ is minimal at $\beta_0 = \beta$.

Replace $S(\beta)$ by the empirical counterpart 
\begin{equation}
S_n(\beta) = \frac{1}{n}\sum_{i = 1}^n\big[Y_i - g(X_i'\beta)\big]^2
\end{equation}

and minimize $S_n(\beta)$ instead of $S(\beta)$.


In the present case, $g$ is in fact unknown and it must be estimated. However, this cannot be done directly with kernel estimation as $\beta_0$ is also unknown. Still, for a given $\beta$ we can estimate

\begin{equation}
G(X_i'\beta) \stackrel{def}{=} E(Yi|X_i'\beta) = E[g(X_i'\beta_0)|X_i'\beta]
\end{equation}
 by the kernel method. 

Ichimura (1993) proposes modifications of the usual kernel estimation to estimate $G(X_i'\beta)$. Markedly, observation $i$ is excluded from the calculation of $G(X_i'\beta)$. Otherwise, consider a relatively small bandwidth is used. Then, $S_n(\beta)$ is trivially minimized when $\hat{G}(X_i'\beta) = Y_i$. By leaving one observation out, this problem is countered. Also, this validates the ability to predict the $i$th observation using the remaining observations in the sample. Therefore, outside the sample prediction is improved. In this way, a leave-one-out Nadaraya-Watson (NW) kernel estimator is obtained

\begin{equation}
\hat{G}_{-i}(X_i'\beta) \equiv \hat{E}_{-i}(Yi|X_i'\beta) = \frac{(nh)^{-1}\sum_{j=1, j \neq i }^{n}  Y_jK(\frac{X_j'\beta - X_i'\beta}{h_n})}{\hat{p}_{-i}(X_i'\beta)},
\end{equation}

where $\hat{p}_{-i}(X_i'\beta) = \frac{1}{nh_n}\sum_{j=1,j \neq i}^{n}K(\frac{X_j'\beta - X_i'\beta}{h_n})$.  

The denominator $\hat{p}(X_i'\beta)$ is random and it becomes necessary to trim small values, particularly at the tails of the distribution. Otherwise, the value for the NW kernel estimator explodes. Let $p(x'\beta)$ denote the probability density function (PDF) of $X_i'\beta$ and $A_\delta$ and $A_n$ be the sets

\[ A_\delta = \{ x : p(x'\beta) \geq \delta, \text{ for all }  \beta \in \mathcal{B} \}
\]

where $\delta > 0$ is a constant, $\mathcal{B} \in \mathbb{R}^q$, and

\[ A_n = \{ x : \norm{x - x^*} \leq 2h_n \text{ for some } x^* \in A_\delta\}.
\]

Then, for $x \in A_\delta$ the denominator does not get too close to zero. The set $A_n$ where $\norm{\cdot}$ is a Euclidean norm, is larger than $A_\delta$ but as $ n \rightarrow \infty $, $h_n \rightarrow 0$, $A_n$ shrinks to $A_\delta$. 

With this in mind, it is possible to choose $\beta$ by using a weighted NLS (WNLS) method

\begin{equation}
S_n(\beta) = \frac{1}{n} \sum_{i=1}^{n}  [Y_i - \hat{G}_{-i}(X_i'\beta)]^2w(x_i)\mathbf{1}{(X_i \in A_n)}
\end{equation}

where $\mathbf{1}{(X_i \in A_n)}$ is a trimming function that ensures that the random denominator is positive with high probability, and $w(x_i)$ is an appropriate nonnegative weighting function to correct heteroskedasticity, as explained in section 4.2. 

To derive the asymptotic distribution of $\hat{\beta}_n$, first certain restrictions must be imposed. In particular, a bounded second order kernel with bounded support must be used and $ h_n = O(n^{-\frac{1}{5}})$.

\newtheorem{theorem}{Theorem}[section]

\begin{theorem}
According to Ichimura (1993), $\hat{\beta}_{n}$ has convergence rate $n^{-\frac{1}{2}}$ and is asymptotically normally distribution as in

%\begin{enumerate}[label=(\roman*)]
	%\item The set $A_\delta$ is compact, and the weight function $w(\cdot)$ is bounded and positive on $A_\delta$. Define the set $D_z = \{ z : z = x'\beta, \beta \in \mathcal{B}$,$ x \in A_\delta\}$. Letting $p(\cdot)$ denote the PDF of $z \in D_z$, $p(\cdot)$ is bounded below by a positive constant for all $z \in D_z$.
	%\item $g(\cdot)$ and $p(\cdot)$ are three times differentiable with respect to $z = x'\beta$. The third derivatives are Lipschitz continuous uniformly over $\mathcal{B}$ for all $z \in D_z$.
	%\item The kernel function is a bounded second order kernel having bounded support, is twice differentiable, and its second derivative is Lipschitz continuous.
	%\item $E|Y^m| < \infty$ for some $m \geq 3$. $cov(Y|x)$ is bounded and bounded away from zero for all $x \in A_\delta$. $ln(h)/[nh^{3 + \frac{3}{m-1}}] \rightarrow 0$ and $nh^8 \rightarrow 0$ as $n \rightarrow \infty $ .
	
%\end{enumerate}


\[ \sqrt{n}(\hat{\beta}_n - \beta_0) \stackrel{d}{\rightarrow} N(0,\Omega_I), \] where $\Omega_I = V^{-1}\Sigma V^{-1}$, $I$ stands for Ichimura
and 
\[\Sigma = E\{w(X_i)^2\sigma^2(X_i)(g_i^{(1)})^2(X_i - E_A(X_i|X_i'\beta_0)) \times (X_i - E_A(X_i|X_i'\beta_0))'\},\]

with $g_i^{(1)} = [\partial g(v)/\partial v]|_{v = X_i'\beta_0}, E_A(X_i|v) = E(X_i|x_A'\beta_0 = v)$ with $x_A$ having the distribution of $X_i$ conditional on $X_i \in A_\delta$, and

\[ V = E[w(X_i)(g_i^{(1)})^2(X_i - E_A(X_i|X_i'\beta_0))(X_i - E_A(X_i|X_i'\beta_0))'].\]

\end{theorem}

Note $\sqrt{n}(\hat{\beta}_n - \beta_0)=O_p(1)$ and $(\hat{\beta}_n - \beta_0) = O_p\left(\frac{1}{\sqrt{n}}\right)$. Therefore, root-n consistency is attained, which is the optimal rate for most parametric methods. Also, $h_n = O(n^{-\frac{1}{5}})$, which is the optimal bandwidth balancing bias and variance for a  nonparametric density.

A consistent estimator of $\Omega_I$ is given by

\[ \hat{\Omega}_I = \hat{V}^{-1}\hat{\Sigma}\hat{V}^{-1}, \]

where $\hat{V} = n^{-1}\sum_{i} w(X_i)(\hat{g}^{(1)})^2(X_i'\hat{\beta}_n)(X_i - \hat{E}(X_i|X_i'\beta_n))(X_i - \hat{E}(X_i|X_i'\beta_n))', \hat{\Sigma} = n^{-1}\sum_{i} w(X_i)^2\hat{\varepsilon}_{i}^{2}(\hat{g}^{(1)})^2(X_i'\hat{\beta}_n)(X_i - \hat{E}(X_i|X_i'\beta_n))', \hat{\varepsilon}_i = Y_i - \hat{g}(X_i'\hat{\beta}_n), \hat{g}^{(1)}(X_i'\hat{\beta}_n) \linebreak
= [\partial \hat{g}_{-i}/\partial \beta]|_{\beta=\hat{\beta}_n}$, $\hat{g}_{-i}(X_i'\hat{\beta}_n)$ is defined in (8), $\hat{E}(X_i|X_i'\beta_n)' = \sum_{j} X_jK((X_i - X_j')'\hat{\beta})/ \sum_{j}K((X_i - X_j)'\hat{\beta}_n).$

For what follows the trimming set $A_\delta$ is ignored and $w(\cdot)$ is set to 1. Furthermore, assume $\beta_n - \beta_0 = O(n^{-\frac{1}{2}})$ and $\hat{\beta}_n - \beta_0 = O_p(n^{-\frac{1}{2}})$. Then, 


\begin{align*}
S_{n}(\beta_n) & = \frac{1}{n}\sum_i \{ Y_i - \hat{G}_{-i}(X_i'\beta_n)\}^2 = \frac{1}{n}\sum_i\{Y_i - \hat{G}_{-i}(X_i'\beta_n) +  \hat{G}_{-i}(X_i'\beta_0) \\
			 & - \hat{G}_{-i}(X_i'\beta_0) \}^2 = \frac{1}{n} \sum_i \{Y_i - G(X_i'\beta_n) + o_p(1) + \hat{G}_{-i}(X_i'\beta_0) \\
			 & - \hat{G}_{-i}(X_i'\beta_0) \}^2 = \frac{1}{n}\sum_i \{ Y_i - G(X_i'\beta_n) + \hat{G}_{-i}(X_i'\beta_0) \\
			 & - g(X_i'\beta_0) + o_p(1) \}^2 = \frac{1}{n} \sum_i \{ g(X_i'\beta_0) + \varepsilon_i - G(X_i'\beta_n) + \hat{G}_{-i}(X_i'\beta_0) \\
			 & - g(X_i'\beta_0) + o_p(1) \}^2 = \frac{1}{n} \sum_i \{ \varepsilon_i + \hat{G}_{-i}(X_i'\beta_0) - E[g(X_i'\beta_0)|X_i'\beta_n] \\
			 & + o_p(1) \}^2 = \frac{1}{n}\sum_i \{ g(X_i'\beta_0) - E[g(X_i'\beta_0)|X_i'\beta_n] +  \varepsilon_i + o_p(1)\}^2 \\
			 & = \frac{1}{n}\sum_i \{ g(X_i'\beta_0) - E[g(X_i'\beta_0)|X_i'\beta_n] +  \varepsilon_i\}^2 + o_p(1)
\end{align*}

Using a Taylor expansion:

\begin{align*}
g(X_i'\beta_0) - E[g(X_i'\beta_0)|X_i'\beta_n)] & = g(X_i'\beta_0) - g(X_i'\beta_n) \\
											 & - g^{(1)}(X_i'\beta_n)E[(\beta_0 - \beta_n)X_i'|X_i'\beta_n] + o_p(1) \\
				                              & = g^{(1)}(X_i'\beta_n)( X_i - E[X_i'|X_i'\beta_n)(\beta_0 - \beta_n) + o_p(1)
\end{align*}


Given that $E(\varepsilon) = 0$, the mean of $\varepsilon_i$ is $o_p(1)$. Then, $\frac{1}{n}\sum_i\varepsilon_io_p(1) = o_p(1)$. Hence, for $g_{i}^{(1)} = g^{(1)}(X_i'\hat{\beta}_n)$ and $v_i = X_i - E[X_i'|X_i'\beta_n]$:

\begin{align*}
S_{n}(\hat{\beta}_n) & = (\beta_0 - \hat{\beta}_n)'\left[\frac{1}{n}\sum_i (g_i^{(1)})^2v_iv_i'\right](\beta_0 - \hat{\beta}_n) \\
             & + 2\frac{1}{n}\sum_i\varepsilon_ig_i^{(1)}v_i'(\beta_0 - \hat{\beta}_n) + \frac{1}{n}\sum_i \varepsilon_i^2 + o_p(1)
\end{align*}

Minimizing in order to $\hat{\beta}_n$ and ignoring terms independent of $\hat{\beta}_n$ and $o_p(1)$:
\[2\frac{1}{n}(\hat{\beta}_n - \beta_0)\sum_i(g_{i}^{(1)})^2v_iv_i' - 2\frac{1}{n}\sum_i\varepsilon_ig_i^{(1)}v_i' = 0 \]
Then, 
\begin{align*}
\sqrt{n}(\hat{\beta}_n - \beta_0) & = (\frac{1}{n}\sum_i(g_{i}^{(1)})^2v_iv_i')^{-1}\frac{1}{\sqrt{n}}\sum_i\varepsilon_i g_{i}^{(1)}v_i \\
     					  & = (\frac{1}{n}\sum_i(g_{i0}^{(1)})^2v_{i0}v_{i0}')^{-1}\frac{1}{\sqrt{n}}\sum_i\varepsilon_i g_{i0}^{(1)}v_{i0} + o_p(1).
\end{align*}

For $g_{i0}^{(1)} = g^{(1)}(X_i'\beta_0)$ and $v_{i0} = X_i - E[X_i'|X_i'\beta_0]$. With this in mind, the Central Limit Theorem (CLT) and Law of Large Numbers (LLN) we obtain the result from Theorem 4.1 if we set $w(X_i) = 1$.


%Recall when we described identiÖcation that we required the dimension of Xi to be 2 or larger.
%Suppose that Xi is one-dimensional.


\subsection{Bandwidth Selection} % (fold)
\label{sub:Bandwidth Selection}

Ichimura (1993) suggests that the smoothing parameter $h$ should be chosen so as to satisfy
 $ln(h_n)/[nh_n^{3 + \frac{3}{v-1}}] \rightarrow 0$ and $nh_n^8 \rightarrow 0$ as $n \rightarrow \infty $, where $v \geq 3$ is a positive integer whose specific values depend on the existence of a number of finite moments  of $Y$ along with the smoothness of the unknown function $g(\cdot)$. This allows for optimal smoothing, that is, $h_n = O(n^{-\frac{1}{5}})$. However, Ichimura (1993) gives a range of bandwidth which enables the construction of a root-$n$ consistent $\hat{\beta}_n$, but excludes the size of bandwidth that is optimal for estimating $g(\cdot)$. \footnote{Hall (1989) shows that two very different bandwidths may be necessary to construct good estimators of both $g(\cdot)$ and $\beta$.}
 
With this in mind, H{\"a}rdle et al. (1993) suggest an empirical way of selecting the bandwidth for optimal smoothing of both $g(\cdot)$ and $\beta$. This can be attained by selecting $h_n$ and $\beta$ simultaneously by minimizing

\begin{equation}
M(\beta, h_n) = \sum_i \left[ Y_i - \hat{G}_{-i}(X_i'\beta, h_n) \right]^2\mathbf{1}{(X_i \in A_\delta)},
\end{equation}

where $\hat{G}_{-i}(X_i'\beta, h_n) = \hat{G}_{-i}(X_i'\beta)$ and $A_\delta$ is the trimming set, as defined above.	

\subsection{Weight Function} % (fold)
\label{sub:Weight Function}

In case heteroskedastic data is used, the model must be changed as it is no longer efficient and standard errors and $t$-statistics are biased.
Heteroskedasticity occurs when the variance of the unobservable error $\varepsilon$, conditional on  the vector of independent variables, is not constant. That is, $Var(\varepsilon_i|X_i) = \sigma_i^2$. In particular, the variance of $\varepsilon$ might be a function of the independent variables.
Moreover, efficiency is desirable because the more efficient an estimator is, the smaller the amount of dispersion it has around its expected value and the more precise it is as an estimator of the corresponding parameter. 

Following the idea of Generalized Least Squares (GLS), the model must then be transformed in order to achieve efficiency. Also analogously to GLS, if the transformation depends upon unknown parameters, these need to be estimated and an analogue of Feasible GLS (FGLS) is obtained instead. On the latter case, a weight function is used that assumes a general form of heteroskedasticy. This weight function, $w$, is chosen so as to maximize the asymptotic efficiency of the estimator. 

%Within the class of weighted nonlinear least squares (WNLS) estimators, an estimator is asymptotically efficient given the covariance matrix $\Omega$ of its asymptotic distributions and the covariance matrix $\Omega^*$ of the other WNLS estimators, if $\Omega^* - \Omega$ is positive semidefinite. Even though the class of all regular estimators of single index models may include estimators that are not semiparametric WNLS estimators, the definition of asymptotically efficient estimator remains the same. 

%Chamberlain (1987) investigated the case for nonlinear regression model (not necessarily a single index model) in which $g$ is known. 

%The model is $E(Y|X = x) = g(x, \beta_0)$. The variance function, $\sigma^2(x) = E\{[Y - g(x, \beta_0)]^2| X = x\}$, is unknown. Chamberlain (1986) showed that the efficiency bound is 

%\[ \Omega_{NLR} = \left\{E\left[\frac{1}{\sigma^2}\frac{\partial g(X, \beta_0)}{\partial \beta} \frac{\partial g(X, \beta_0)}{\partial \beta'}\right]\right\}^{-1} \]

%for a WNLS estimator of $\beta_0$ with weight function $w(x) = \frac{1}{\sigma^2(x)}$. 

The problem of efficient estimation of $\beta_0$ in a single index model with unknown $g(\cdot)$ is analyzed by Hall and Ichimura (1991) and Newey and Stocker (1993). Under certain regularity conditions, the efficiency bound for the single index model, with unknown $g(\cdot)$ and using only data for which $X \in A_{\delta}$, is $\Omega_I$ from Theorem 4.1., for $w(x) = \frac{1}{\sigma^2(x)}$.\footnote{The assumption that only data where  $X \in A_{\delta}$ is used can be relaxed by letting $A_\delta$ grow very slowly as $n$ increases} Thus, the weight function allows us to weight each observation by a factor proportional to the error variance. Moreover, observations with higher variance get a smaller weight. 

The efficiency bound is then

\begin{equation}
\Omega_{SI} = \left\{ E\left[\frac{1}{\sigma^2(x)}\frac{\partial}{\partial \beta}
 G(X'\beta,\beta)\frac{\partial}{\partial \beta} G(X'\beta,\beta) \right] \right\}^{-1}
\end{equation}

where $SI$ stands for single index and, considering the result shown above, we have
\begin{align*}
g(X_i'\beta_0) - E_A[g(X_i'\beta_0)|X_i'\beta)] & = g^{(1)}(X_i'\beta)( X_i - E_A[X_i'|X_i'\beta])(\beta_0 - \beta) \\
											   & + O(n^{-1})
\end{align*}
for $E_A(X_i|v) = E(X_i|x_A'\beta_0 = v)$ with $x_A$ having the distribution of $X_i$ conditional on $X_i \in A_\delta$. Minimizing in order to $\beta$ and ignoring terms independent of $\beta$ and $O(n^{-1})$
\begin{align*}
 \frac{\partial}{\partial \beta} G(X_i'\beta) & = g^{(1)}(X_i'\beta)( X_i - E_A[X_i'|X_i'\beta]) \\
 										   & = g^{(1)}(X_i'\beta_0)( X_i - E_A[X_i'|X_i'\beta]) + o_p(1).
\end{align*}

This bound is achieved by the semiparametric WNLS estimator if $\sigma^2(X)$ is known or independent of X. 

When $\sigma^2(X)$ is unknown, following the reasoning above, the most efficient estimator is an analogue of FGLS.  Consider a given consistent estimator, say $s_{n}^{2}(x)$. If $w(X) = \frac{1}{s_{n}^{2}(x)}$ in the semiparametric WNLS estimator, the asymptotic efficiency bound will be reached. Thus, an asymptotic efficient estimator of $\beta_0$ is obtained even when $\sigma^2(X)$ is unknown. This consistent estimator uses a two-step procedure. On the first step, minimize function (9) with respect to $\beta$ for $w(x) = 1$. We obtain an estimator $\hat{\beta}_n$ that is $n^{\frac{1}{2}}$-consistent and asymptotically normal but inefficient. This estimator is used on the second step to calculate the weight function $\hat{w}_i(x) = \frac{1}{\hat{\sigma}_{i}^{2}}$. Here, $\hat{\sigma}_{i}^{2}$ is defined by Robinson (1987) as 
\[\hat{\sigma}_{i}^{2} = \frac{1}{k}\sum_{j=1}^{n} \mathbf{1}{(x_j \in N_k(x_i))}\hat{\varepsilon}_{j}^{2} ,\]
%The assumption that $X \in A_{\delta}$ can be replaced by the assumption that $A_{\delta}$ grows very slowly as $n$ increases.
and $\hat{\varepsilon}_i = Y_i - \hat{G}_{-i}(X_i'\hat{\beta_n})$. Further, $N_k(x_i)$ is the set of $k$ observations of $x_j$ closest to $x_i$ in weighted Euclidean norm.



\subsection{Model's disadvantages} % (fold)
\label{sub:Model's disadvantages}


it is important to realize that the minimization of a nonlinear objective function such as equation (9) might be computationally costly. The WNLS estimator is computed by iterative methods. Start with an initial random guess for the estimator $\hat{\beta}_n^{1}$ such as $\hat{\beta}_n^{1} = - \frac{1}{n}\sum_i y_i\hat{f'}(x_i)$ where $f'$ is the derivative of the density of $x_i$ and $\hat{\beta}_n^{1}$ follows the restrictions from section 3.\footnote{This estimator is a density weighted average derivative estimator such as in Stoker (1986).} Attain the kernel estimate $\hat{G}_{-i}(X_i'\hat{\beta}_n^{1})$ and thus $S_n(\hat{\beta}_n^{1})$. Perturb $\hat{\beta}_n^{1}$ to obtain $\frac{\partial S_n(\beta)}{\partial\beta} |_{\hat{\beta}_n^{1}}$ and update $\hat{\beta}_n^{2} = \hat{\beta}_n^{1} + A_n  \frac{ \partial S_n(\beta)}{\partial\beta}|_{\hat{\beta}_n^{1}}$ where $A_n$ is the size of the random disturbance. This process should be repeated until convergence. Yet, this is computationally difficult, specially because there might be multiple local minima, in case, for example, the objective function is multimodal or nonconvex. 

One may consider  as an alternative a direct estimation method that does not require optimization of problems involving iterative solutions.  H{\"a}rdle and Stoker (1989) provide such a method. However, according to Racine and Li (2007), for small sample settings, H{\"a}rdle and Stoker's (1989) direct method may be less appealing than Ichimura's (1993) iterative method, as the curse of dimensionality inherited from its first stage might not disappear on the second stage.

\section{Klein and Spady's binary estimation model} % (fold)
\label{sec:section_about_references_within_the_document}
In this section we analyze Klein and Spady's (1993) semiparametric binary choice model, used to estimate model (2) when $Y \in \{0,1\}$.   This method exhibits $n^{-\frac{1}{2}}$ efficiency, asymptotic normality and asymptotic efficiency. 

The model is defined as

\begin{equation}
Y_i =  \mathbf{1}{(X_i'\beta \geq \varepsilon_i)},
\end{equation}
where $\varepsilon$ is a random disturbance. Similarly to Ichimura, let $g(\cdot)$ denote the distribution of $\varepsilon_i$. Then, $G(X_i'\beta) = E(g(X_i'\beta_0)|X_i'\beta)$. For known $g(\cdot)$, the asymptotically efficient estimator of $\beta_0$ is a maximum likelihood estimator (MLE). 

The $log-likelihood$ function is

\begin{equation}
\mathcal{L}_n(\beta) = \frac{1}{n}\sum_{i=1}^n \left\{ (1 - Y_i)ln[ 1 - g(X_i'\beta)] + Y_iln[g(X_i'\beta)] \right\},
\end{equation}

where

\begin{enumerate}[label=(\roman*)]
		\item Y is binomial with realizations 1 and 0, X is a vector of exogenous variables and $\beta$ is an unknown parameter vector;
		\item $\{x_i,y_i\}$ for i = 1, ..., n is an i.i.d. sample.
\end{enumerate}

It is clear from equation (13) that restrictions must be imposed such that any estimate of $g(\cdot)$ is kept sufficiently far away from 0 and 1. This can be achieved by using a simplified trimming function $\tau_i = \mathbf{1}{(X_i \in A_\delta)}$ that restricts X to a fixed set $A_\delta$ on which $G$ is bounded away from 0 and 1.

Similarly to Ichimura's method, Klein and Spady (1993) suggest replacing $g(\cdot)$ with the leave-one-out NW from equation (8). Again, we leave one observation out as otherwise, for a relatively small bandwidth, the objective function would be maximized by setting $G(X_i'\beta)=0$ when $Y_i=0$ and $G(X_i'\beta)=1$ when $Y_i=1$. That is, the objective function would take the value zero. By leaving one observation out, outside the sample prediction is improved. 

With this in mind, the maximization problem takes the form

\begin{equation}
\mathcal{L}_n(\beta) = \frac{1}{n}\sum_{i=1}^n \tau_{i}\{ (1 - Y_i)ln[ 1 - \hat{G}_{-i}(X_i'\beta)] +  Y_iln[\hat{G}_{-i}(X_i'\beta)]\}.
\end{equation}

To derive the asymptotic distribution of $\hat{\beta}_n$, certain assumptions are required in terms of bandwidth and kernel choice. In particular, Klein and Spady (1993) use an higher order kernel. Higher order kernels allow for increased smoothness and faster rates of convergence. However, for these desirable properties to ``kick in'', a high number of observations might be necessary. \footnote{Marron (1992) gives a particular example in which at least 10 000 observations are necessary for the a fourth order kernel to become decidedly dominant relatively to a nonnegative second order kernel.} An example of a higher order kernel is
\[\int z^{2}K(z)dz = 0.\]
Furthermore, the bandwidth must satisfy the rate $ n^{-\frac{1}{6}} < h_n < n^{-\frac{1}{8}}$.


\begin{theorem}
According to Klein and Spady (1993), $\hat{\beta}_{n}$ is $n^{-\frac{1}{2}}$ and has asymptotic normal distribution as in 

\[\sqrt{n}(\hat{\beta}_{n} - \beta_0) \stackrel{d}{\rightarrow} N(0,\Omega_{KS}),
\]
where \[ \Omega_{KS} = \left\{ E\left[\frac{\partial}{\partial \beta}
 G(X_i'\beta)\frac{\partial}{\partial \beta} G(X_i'\beta)'\frac{1}{g(X_i'\beta_0)(1 - g(X_i'\beta_0))} \right]\right\}^{-1}. \]
 
$KS$ stands for Klein and Spady and $\Omega_{KS} = \Omega_{SI}$, i.e., the estimator is asymptotically efficient.

\end{theorem}

To have a good grasp of this result, it is fundamental to understand the expression for $\Omega_{KS}$. In particular, if one is to show that $\Omega_{KS} = \Omega_{SI}$, it is necessary to understand that $\frac{1}{g(X_i'\beta_0)(1 - g(X_i'\beta_0))}$ indeed corresponds to $\frac{1}{\sigma^2(x)}$ in equation (10). 

It is known that for a binary choice model, 
\[Var(Y|X = x) = P(Y = 1|X = x)[1 - P(Y = 1|X = x)].\]
Thus, in the present case
\begin{align*}
Var(Y|X = x) & = E[\mathbf{1}{(Y = 1)}|X = x]\{1 - E[\mathbf{1}{(Y = 1)}|X = x]\} \\
&=E[\mathbf{1}{(x'\beta_0 \geq \varepsilon)}|X = x]\{1 - E[\mathbf{1}{(x'\beta_0 \geq \varepsilon)}|X = x]\} \\ 
& = g(x'\beta_0)[1 - g(x'\beta_0)]
\end{align*}

where the last equality follows from the assumption $E[\varepsilon|X]=0$.

Consider now a semiparametric WNLS estimation of $\beta_0$ such as in Ichimura's (1993) model. Differentiate the right-hand side of equation (14) with respect to $\beta_n$. Then, the first order condition is
\begin{equation}
\frac{1}{n} \sum_{i=1}^n \frac{\frac{\partial\hat{G}_{-i}(X_i'\beta_n)}{\partial \beta_n}}{\hat{G}_{-i}(X_i'\beta_n)(1 - \hat{G}_{-i}(X_i'\beta_n))} (Y_i - \hat{G}_{-i}(X_i'\beta_n)) X_i \tau_i = 0
\end{equation} 

with probability approaching 1 as $n \rightarrow \infty$. This is the same as the first-order condition for semiparametric WNLS estimation of $\beta_0$ with the weight function

\begin{align*}
w_i & = \{ \hat{G}_{-i}(X_i'\beta_n)[ 1 - \hat{G}_{-i}(X_i'\beta_n)]\}^{-1} = \{ G(X_i'\beta_n)[ 1 - G(X_i'\beta_n)]\}^{-1} + o_p(1) \\
     & = \{ G(X_i'\beta_0)[ 1 - G(X_i'\beta_0)]\}^{-1} + o_p(1).
\end{align*}


Seeing to section 4.2., it follows that semiparametric maximum-likelihood estimator of $\beta_0$ in Klein and Spady's model is asymptotically efficient.\footnote{This is only true for ``first order efficiency'', i.e., when $E(X_i|X_i'\beta)$ is linear in $X_i'\beta$.}

\subsection{Bandwidth Selection} % (fold)
\label{sub:Bandwidth Selection}

Klein and Spady (1993) do not discuss the choice of bandwidth, apart from the requirement that $ n^{-\frac{1}{6}} < h_n < n^{-\frac{1}{8}}$. However, similarly to Ichimura's (1993) method and following H{\"a}rdle et al. (1993), an empirical way of selecting the bandwidth for optimal smoothing of both $g(\cdot)$ and $\beta$ can be conjectured. This is attained by choosing $h_n$ jointly with $\beta$ when maximizing
\begin{equation}
M_n(\beta, h_n) = \frac{1}{n}\sum_{i=1}^n \tau_{i}\{ (1 - Y_i)ln[ 1 - \hat{G}_{-i}(X_i'\beta, h_n)] +  Y_iln[\hat{G}_{-i}(X_i'\beta, h_n)]\}.
\end{equation}

\subsection{Model's disadvantages} % (fold)
\label{sub:Model's disadvantages}

Analogously to Ichimura's (1993) model, optimization of the maximum likelihood function, using the iteration between the computation of $\hat{\beta}_n$ given $\hat{G}_{-i}$ and the computation of $\hat{G}_{-i}$ given $\hat{\beta}_n$, is recurrently difficult.  Especially so when the optimization leads to multiple local maxima. Such an event could be explained by an objective function that is either multimodal or nonconcave. 

Zhou and Lang (1995) provide an alternative using an ``easy to compute'' semiparametric estimator for binary choice models. The method is based on a semiparametric interpretation of the Expectation and Maximization (EM) principle (Dempster et al (1977)) and the least squares approach. It preserves root-n consistency and is asymptotically normally distributed, but has the novelty of being fast and easy to compute.
% section section_about_references_within_the_document (end)

\section{Comparing Ichimura's and Klein and Spady's model: a theoretical perspective} % (fold)
\label{sec:Comparing Ichimura's and Klein and Spady's model: a theoretical perspective}

Klein and Spady's (1993) model seems more adequate than Ichimura's (1993) for the binary choice model case. Klein and Spady's (1993) model is fully efficient in the sense that it reaches the semiparametric efficiency bound. In fact, under certain assumptions, such as asymptotic normality and root-n consistency, if the maximum likelihood solution exists, the maximum likelihood estimate is the most efficient estimator of $\beta_0$ (Eliason (1993)). As Ichimura's (1993) and Klein and Spady's models (1993) satisfy both these conditions, the previous statement seems to give support to our claim. Maximum-likelihood functions have ``natural'' weighting. On the other hand, Ichimura's (1993) model requires a weight function, and potentially the two-step procedure described in section 4.2., to reach the asymptotic bound. Thus, much is saved in terms of simplicity and efficiency is in general improved by using Klein and Spady's (1993) model for the binary case.
\newpage 

\bibliographystyle{natdin}
	\bibliography{references} % expects file "references.bib"
	\addcontentsline{toc}{section}{References}
\end{document}