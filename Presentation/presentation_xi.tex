%%%%%%%%%%%%%%%%%%%%%%%%%%%%%%%%%%%%%%%%%
% Beamer Presentation
% LaTeX Template
% Version 1.0 (10/11/12)
%
% This template has been downloaded from:
% http://www.LaTeXTemplates.com
%
% License:
% CC BY-NC-SA 3.0 (http://creativecommons.org/licenses/by-nc-sa/3.0/)
%
%%%%%%%%%%%%%%%%%%%%%%%%%%%%%%%%%%%%%%%%%

%----------------------------------------------------------------------------------------
%	PACKAGES AND THEMES
%----------------------------------------------------------------------------------------

\documentclass{beamer}

\mode<presentation> {

% The Beamer class comes with a number of default slide themes
% which change the colors and layouts of slides. Below this is a list
% of all the themes, uncomment each in turn to see what they look like.

%\usetheme{default}
%\usetheme{AnnArbor}
%\usetheme{Antibes}
%\usetheme{Bergen}
%\usetheme{Berkeley}
%\usetheme{Berlin}
%\usetheme{Boadilla}
%\usetheme{CambridgeUS}
%\usetheme{Copenhagen}
%\usetheme{Darmstadt}
%\usetheme{Dresden}
%\usetheme{Frankfurt}
%\usetheme{Goettingen}
%\usetheme{Hannover}
%\usetheme{Ilmenau}
%\usetheme{JuanLesPins}
%\usetheme{Luebeck}
%\usetheme{Madrid}
%\usetheme{Malmoe}
%\usetheme{Marburg}
%\usetheme{Montpellier}
%\usetheme{PaloAlto}
%\usetheme{Pittsburgh}
%\usetheme{Rochester}
\usetheme{Singapore}
%\usetheme{Szeged}
%\usetheme{Warsaw}

% As well as themes, the Beamer class has a number of color themes
% for any slide theme. Uncomment each of these in turn to see how it
% changes the colors of your current slide theme.

%\usecolortheme{albatross}
%\usecolortheme{beaver}
%\usecolortheme{beetle}
%\usecolortheme{crane}
%\usecolortheme{dolphin}
%\usecolortheme{dove}
%\usecolortheme{fly}
%\usecolortheme{lily}
%\usecolortheme{orchid}
%\usecolortheme{rose}
%\usecolortheme{seagull}
%\usecolortheme{seahorse}
%\usecolortheme{whale}
%\usecolortheme{wolverine}

%\setbeamertemplate{footline} % To remove the footer line in all slides uncomment this line
%\setbeamertemplate{footline}[page number] % To replace the footer line in all slides with a simple slide count uncomment this line

%\setbeamertemplate{navigation symbols}{} % To remove the navigation symbols from the bottom of all slides uncomment this line
}

\usepackage{graphicx} % Allows including images
\usepackage{booktabs} % Allows the use of \toprule, \midrule and \bottomrule in tables
\usepackage[backend=bibtex, natbib=true, bibencoding=inputenc, bibstyle=authoryear-ibid, citestyle=authoryear-comp, maxcitenames=3, maxbibnames=10]{biblatex}
\setlength{\bibitemsep}{1.5ex}
\addbibresource{project_refs.bib}

%----------------------------------------------------------------------------------------
%	TITLE PAGE
%----------------------------------------------------------------------------------------


\title[Ichimura's and Klein and Spady's methods]{Semiparametric Single Index Models: Ichimura and Klein and Spady's methods} % The short title appears at the bottom of every slide, the full title is only on the title page

\begin{document}


\author{Isa Marques, Xi Sun, Xueying Liu} % Your name
{
%{\small UUU}\\[1ex]
%{\bf Add other authors}\\
%{\small and affilitations}\\[1ex]
}

\date{
{\small Bonn, Germany}\\
{\small \today}
}

\begin{frame}
\titlepage % Print the title page as the first slide
\end{frame}


%----------------------------------------------------------------------------------------
%	PRESENTATION SLIDES
%----------------------------------------------------------------------------------------

%------------------------------------------------
%\section{First Section} % Sections can be created in order to organize your presentation into discrete blocks, all sections and subsections are automatically printed in the table of contents as an overview of the talk
%------------------------------------------------

%\subsection{Subsection Example} % A subsection can be created just before a set of slides with a common theme to further break down your presentation into chunks

\begin{frame}[t]
    \frametitle{Outline}
    
    \begin{itemize}
        \item Introduction
        \item Identification
        \item Ichimura's (1993) method
        \item Klein and Spady's (1993) method
        \item Monte Carlo simulation
    \end{itemize}
    \note{~}
\end{frame}


\begin{frame}[t]
    \frametitle{Introduction}
	 \textbf{Definition of Single Index Model:} 
	 (following notation in \citet{LiRacine07})
 
	 \begin{eqnarray}
		Y = g(X'\beta) + \epsilon,  % with i =1,2 ..., n?, iid?
 	 \end{eqnarray}

		\begin{enumerate}
			\item $\{x_i,y_i\}$ for i = 1, ..., n is an i.i.d. sample;
			\item Y is the dependent variable, $X\in \mathbb{R}^{q}$ is a vector of explanatory variables, $\beta$ is the q $\times$ 1 						vector of unknown parameters; 
			\item The functional form of the linear index, $X'\beta$ is specified and it is a single index because it is a scalar;
			\item $g: \mathbb{R} \rightarrow \mathbb{R} $ is unspecified; % dimensions?
			\item The conditional probability of $\epsilon$ conditioned on X is not specified except $ E(\epsilon|x) = 0 $;
	   \end{enumerate}

\note{~}      
\end{frame}

\begin{frame}[t]
	\frametitle{Introduction}
	\textbf{Advantage of Single Index Model vs. Parametric and Nonparametric Models}
	\begin{itemize}
	\item Parametric model: prespecifies functional forms which is assumed to be fully described by a finite set of parameters.
			
			eg, Binary choice model: 
			\[E(Y|x) = 1 - F(-(\alpha + x'\beta))\]
			Misspecification of error distribution leads to inconsistent estimation of parameters.
			
	\item Fully nonparametric models: \[Y = g(X, \epsilon)\ or\ Y = g(X) + \epsilon\]
	      Curse of dimensionality: as the dimensions of the model increase, the convergence rate of the estimator decreases.
	
	\end{itemize}
	

\note{~}
\end{frame}


\begin{frame}[t]
    \frametitle{Identification}
	  Identification(\citet{Horowitz09}): $\beta$ and $ g(\cdot)$ must be uniquely determined by sample data.
     \begin{equation}
		E(Y|x) = g(x'\beta_0).
	  \end{equation}
    \newtheorem{prop1}{Proposition}[section]   
	  \begin{prop1}[Identification of a Single Index Model] 
		
		\begin{enumerate}
        \item  x should not contain a constant and it must contain at least one continuous variable with nonzero coefficient. Furthermore, one component of $\beta_0$ is set to 1.
        \item  The support of $x'\beta_0$ is bounded convex set with at least one interior point. $g$ is differentiable and it is not a constant function on the support of $x'\beta_0$.
        \item  For the discrete components of $x$, varying the values of the discrete variables will not divide the support of $x'\beta_0$ into disjoint subsets, and $g$ must be nonperiodic.		
		\end{enumerate}
    
    \end{prop1}

\note{~}
\end{frame}


\begin{frame}[t]
    \frametitle{Identification}
     %here \citet{Horowitz09}
     \begin{equation*}
		E(Y|x) = g(x'\beta_0).
	  \end{equation*}
    \newtheorem{prop2}{Proposition}[section]   
	  \begin{prop2}[Identification of a Single Index Model] 
		\begin{itemize}
		  \item 1. x should not contain a constant and it must contain at least one continuous variable with nonzero coefficient. Furthermore, one component of $\beta_0$ is set to 1.
		\end{itemize}
    \end{prop2}
    Intuition:
    \begin{itemize}
      
      \item $g^{*}(\gamma + \delta v) = g(v)$, for all $v$ in the support of $X'\beta$.
      
      	 restriction on $\gamma$: location normalization,
      	 
      	 restriction on $\delta$: scale normalization.
      \item Suppose finitely many v in support, there may exist infitely many satisfying $g(\cdot)$, indistinguishable from each other.
      
    \end{itemize}
    

\note{~}
\end{frame}

%------------------------------------------------

\begin{frame}[t]
    \frametitle{Identification}
     %here \citet{Horowitz09}
     \begin{equation*}
		E(Y|x) = g(x'\beta_0).
	  \end{equation*}
    \newtheorem{prop3}{Proposition}[section]   
	  \begin{prop3}[Identification of a Single Index Model] 
		\begin{itemize}
        \item 3. For the discrete components of $x$, varying the values of the discrete variables will not divide the support of $x'\beta_0$ into disjoint subsets, and $g$ must be nonperiodic.
		\end{itemize}
    \end{prop3}
    Intuition:
Assume a continuous $X_1$ with support $\big[0,1\big]$, and a discrete $X_2$ with support $\{0,1\}$, $g$ is strictly increasing and non periodic and set $\beta_1 = 1$ as a \textit{scale normalization}.
    \[
\begin{split}
E[Y| X = (x_1,0)]& = g(x_1), \text{support\ of } g(\cdot): [0,1];  \\
E[Y| X = (x_1,1)]& = g(x_1+\beta_2), \text{support\ of } g(\cdot): [\beta_2,1+\beta_2].
\end{split}
\]

\note{~}
\end{frame}


% Print black screen only in presentation mode for finishing up.
\mode<beamer> {
    \beamersetaveragebackground{black}
    \begin{frame}
        \frametitle{}
    \end{frame}

    \beamersetaveragebackground{white}
}

\begin{frame}[allowframebreaks]
    \frametitle{References}
    \renewcommand{\bibfont}{\normalfont\footnotesize}
    \printbibliography
\end{frame}

%------------------------------------------------

\begin{frame}
\Huge{\centerline{The End}}
\end{frame}

%----------------------------------------------------------------------------------------

\end{document} 