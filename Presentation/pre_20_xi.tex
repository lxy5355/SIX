\documentclass[11pt]{beamer}
% \documentclass[11pt,handout]{beamer}
\usepackage[T1]{fontenc}
\usepackage[utf8]{inputenc}
\usepackage{float, afterpage, rotating, graphicx}
\usepackage{epstopdf}
\usepackage{longtable, booktabs, tabularx}
\usepackage{fancyvrb, moreverb, relsize}
\usepackage{eurosym, calc, chngcntr}
\usepackage{amsmath, amssymb, amsfonts, amsthm, bm} 

\usepackage[backend=bibtex, natbib=true, bibencoding=inputenc, bibstyle=authoryear-ibid, citestyle=authoryear-comp, maxcitenames=3, maxbibnames=10]{biblatex}
\setlength{\bibitemsep}{1.5ex}
\addbibresource{project_refs.bib}

\hypersetup{colorlinks=true, linkcolor=black, anchorcolor=black, citecolor=black, filecolor=black, menucolor=black, runcolor=black, urlcolor=black}

\setbeamertemplate{footline}[frame number]
\setbeamertemplate{navigation symbols}{}
\setbeamertemplate{frametitle}{\centering\vspace{1ex}\insertframetitle\par}


\begin{document}

\title{Topic 7: SEMIPARAMETRIC SINGLE INDEX MODELS: ICHIMURA'S METHOD}

\author[Isa Marques, Xi Sun, Xueying Liu] % \& others]
{
{\bf Project Module Econometrics and Statistics}\\
{\small Isa Marques, Xi Sun, Xueying Liu}\\[1ex]
%{\small UUU}\\[1ex]
%{\bf Add other authors}\\
%{\small and affilitations}\\[1ex]
}

\date{
{\small Bonn, Germany}\\
{\small \today}
}


\begin{frame}
    \titlepage
    \note{~}
\end{frame}

\begin{frame}[t]
    \frametitle{Structure}
    \begin{itemize}
        \item Introduction
        \item Identification
        \item Ichimura's (1993) method
        \item Klein and Spady's (1993) method
        \item Monte Carlo simulation
    \end{itemize}
    \note{~}
\end{frame}


\begin{frame}[t]
    \frametitle{Introduction}
	 \textbf{Definition of Single Index Model:} 
	 (following notation in \citet{LiRacine07})
 
	 \begin{eqnarray}
		Y = g(X'\beta) + \epsilon,  % with i =1,2 ..., n?, iid?
 	 \end{eqnarray}

		\begin{enumerate}
			\item $\{x_i,y_i\}$ for i = 1, ..., n is an i.i.d. sample;
			\item Y is the dependent variable, $X\in \mathbb{R}^{q}$ is a vector of explanatory variables, $\beta$ is the q $\times$ 1 						vector of unknown parameters; 
			\item The functional form of the linear index, $X'\beta$ is specified and it is a single index because it is a scalar;
			\item $g: \mathbb{R} \rightarrow \mathbb{R} $ is unspecified; % dimensions?
			\item The conditional probability of $\epsilon$ conditioned on X is not specified except $ E(\epsilon|x) = 0 $;
	   \end{enumerate}

\note{~}      
\end{frame}

\begin{frame}[t]
	\frametitle{Introduction}
	\textbf{Advantage of Single Index Model vs. Parametric and Nonparametric Models}
	\begin{itemize}
	\item Parametric model: prespecifies functional forms which is assumed to be fully described by a finite set of parameters.
			
			eg, Binary choice model: 
			\[E(Y|x) = 1 - F(-(\alpha + x'\beta))\]
			Misspecification of error distribution leads to inconsistent estimation of parameters.
			
	\item Fully nonparametric models: \[Y = g(X, \epsilon)\ or\ Y = g(X) + \epsilon\].
	      Curse of dimensionality: as the dimensions of the model increase, the convergence rate of the estimator decreases.
	
	\end{itemize}
	

\note{~}
\end{frame}


\begin{frame}[t]
    \frametitle{Identification}
	  Identification(\citet{Horowitz09}): $\beta$ and $ g(\cdot)$ must be uniquely determined by sample data.
     \begin{equation}
		E(Y|x) = g(x'\beta_0).
	  \end{equation}
    \newtheorem{prop1}{Proposition}[section]   
	  \begin{prop1}[Identification of a Single Index Model] 
		
		\begin{itemize}
        \item 1. x should not contain a constant and it must contain at least one continuous variable with nonzero coefficient. Furthermore, one component of $\beta_0$ is set to 1.
        \item 2. The support of $x'\beta_0$ is bounded convex set with at least one interior point. $g$ is differentiable and it is not a constant function on the support of $x'\beta_0$.		
		\end{itemize}
    
    \end{prop1}
	Intuition:
\note{~}
\end{frame}


\begin{frame}[t]
    \frametitle{Identification}
     %here \citet{Horowitz09}
     \begin{equation}
		E(Y|x) = g(x'\beta_0).
	  \end{equation}
    \newtheorem{prop2}{Proposition}[section]   
	  \begin{prop2}[Identification of a Single Index Model] 
		\begin{itemize}
		  \item 3. For the discrete components of $x$, varying the values of the discrete variables will not divide the support of $x'\beta_0$ into disjoint subsets, and $g$ must be nonperiodic.
		\end{itemize}
    \end{prop2}
    Intuition:

\note{~}
\end{frame}

% Print black screen only in presentation mode for finishing up.
\mode<beamer> {
    \beamersetaveragebackground{black}
    \begin{frame}
        \frametitle{}
    \end{frame}

    \beamersetaveragebackground{white}
}

\begin{frame}[allowframebreaks]
    \frametitle{References}
    \renewcommand{\bibfont}{\normalfont\footnotesize}
    \printbibliography
\end{frame}

\end{document}
