%%%%%%%%%%%%%%%%%%%%%%%%%%%%%%%%%%%%%%%%%
% Beamer Presentation
% LaTeX Template
% Version 1.0 (10/11/12)
%
% This template has been downloaded from:
% http://www.LaTeXTemplates.com
%
% License:
% CC BY-NC-SA 3.0 (http://creativecommons.org/licenses/by-nc-sa/3.0/)
%
%%%%%%%%%%%%%%%%%%%%%%%%%%%%%%%%%%%%%%%%%

%----------------------------------------------------------------------------------------
%	PACKAGES AND THEMES
%----------------------------------------------------------------------------------------

\documentclass{beamer}

\mode<presentation> {

% The Beamer class comes with a number of default slide themes
% which change the colors and layouts of slides. Below this is a list
% of all the themes, uncomment each in turn to see what they look like.

%\usetheme{default}
%\usetheme{AnnArbor}
%\usetheme{Antibes}
%\usetheme{Bergen}
%\usetheme{Berkeley}
%\usetheme{Berlin}
%\usetheme{Boadilla}
%\usetheme{CambridgeUS}
%\usetheme{Copenhagen}
%\usetheme{Darmstadt}
%\usetheme{Dresden}
%\usetheme{Frankfurt}
%\usetheme{Goettingen}
%\usetheme{Hannover}
%\usetheme{Ilmenau}
%\usetheme{JuanLesPins}
%\usetheme{Luebeck}
\usetheme{Madrid}
%\usetheme{Malmoe}
%\usetheme{Marburg}
%\usetheme{Montpellier}
%\usetheme{PaloAlto}
%\usetheme{Pittsburgh}
%\usetheme{Rochester}
%\usetheme{Singapore}
%\usetheme{Szeged}
%\usetheme{Warsaw}

% As well as themes, the Beamer class has a number of color themes
% for any slide theme. Uncomment each of these in turn to see how it
% changes the colors of your current slide theme.

%\usecolortheme{albatross}
%\usecolortheme{beaver}
%\usecolortheme{beetle}
%\usecolortheme{crane}
%\usecolortheme{dolphin}
%\usecolortheme{dove}
%\usecolortheme{fly}
%\usecolortheme{lily}
%\usecolortheme{orchid}
%\usecolortheme{rose}
%\usecolortheme{seagull}
%\usecolortheme{seahorse}
%\usecolortheme{whale}
%\usecolortheme{wolverine}

%\setbeamertemplate{footline} % To remove the footer line in all slides uncomment this line
%\setbeamertemplate{footline}[page number] % To replace the footer line in all slides with a simple slide count uncomment this line

%\setbeamertemplate{navigation symbols}{} % To remove the navigation symbols from the bottom of all slides uncomment this line
}

\usepackage{graphicx} % Allows including images
\usepackage{booktabs} % Allows the use of \toprule, \midrule and \bottomrule in tables

%----------------------------------------------------------------------------------------
%	TITLE PAGE
%----------------------------------------------------------------------------------------

\title[Semiparametrics Single Index Models]{Semiparametrics Single Index Models: Ichimura and Klein and Spady's methods} % The short title appears at the bottom of every slide, the full title is only on the title page

\author{Isa, Xi, Xueying} % Your name
\institute[University of Bonn] % Your institution as it will appear on the bottom of every slide, may be shorthand to save space
{
Project Module Econometrics and Statistics \\ % Your institution for the title page
\medskip
\textit{University of Bonn} % Your email address
}
\date{\today} % Date, can be changed to a custom date

\begin{document}

\begin{frame}
\titlepage % Print the title page as the first slide
\end{frame}

\begin{frame}
\frametitle{Overview} % Table of contents slide, comment this block out to remove it
\tableofcontents % Throughout your presentation, if you choose to use \section{} and \subsection{} commands, these will automatically be printed on this slide as an overview of your presentation
\end{frame}

%----------------------------------------------------------------------------------------
%	PRESENTATION SLIDES
%----------------------------------------------------------------------------------------

%------------------------------------------------
\section{First Section} % Sections can be created in order to organize your presentation into discrete blocks, all sections and subsections are automatically printed in the table of contents as an overview of the talk
%------------------------------------------------

\subsection{Subsection Example} % A subsection can be created just before a set of slides with a common theme to further break down your presentation into chunks


%------------------------------------------------

\begin{frame}
\frametitle{Ichimura's (1993) method}
For known $g$, a nonlinear least squares (NLS) method can be used to estimate $\beta_0$ by minimizing:
\begin{equation}
S_n(\beta) = \frac{1}{n}\sum_{i = 1}^n\big[Y_i - g(X_i'\beta)\big]^2
\end{equation}
with respect to $\beta$. 
However, both $g$ and $\beta_0$ are unknown. For a given $\beta$ we estimate instead:
\begin{equation}
G(X_i'\beta) \stackrel{def}{=} E(Yi|X_i'\beta) = E[g(X_i'\beta_0)|X_i'\beta].
\end{equation}
Given this, the weighted NLS problem is as follows:
\begin{equation}
S_n(\beta) = \frac{1}{n} \sum_{i=1}^{n}  [Y_i - \hat{G}_{-i}(X_i'\beta)]^2w(x_i)\mathbf{1}{(X_i \in A_n)}
\end{equation}
where $\hat{G}_{-i}(X_i'\beta)$ is a leave-one-out Nadaraya-Watson kernel estimator,  $\mathbf{1}{(Xi \in A_n)}$ is a trimming function and $w(x_i)$ is a weighting function.
\end{frame}



%------------------------------------------------

\begin{frame}
\frametitle{Ichimura's (1993) method}
\begin{block}{Theorem Ichimura}
{\footnotesize \[ \sqrt{n}(\hat{\beta}_n - \beta_0) \stackrel{d}{\rightarrow} N(0,\Omega_I), \] where $\Omega_I = V^{-1}\Sigma V^{-1}$, and \[\Sigma = E\{w(X_i)^2\sigma^2(X_i)(g_i^{(1)})^2(X_i - E_A(X_i|X_i'\beta_0)) \times (X_i - E_A(X_i|X_i'\beta_0))'\},\] with $g_i^{(1)} = [\partial g(v)/\partial v]|_{v = X_i'\beta_0}, E_A(X_i|v) = E(X_i|x_A'\beta_0 = v)$ with $x_A$ having the distribution of $X_i$ conditional on $Xi \in A_\delta$, and \[ V = E[w(X_i)(g_i^{(1)})^2(X_i - E_A(X_i|X_i'\beta_0))(X_i - E_A(X_i|X_i'\beta_0))'].\] \par}
\end{block}

{\footnotesize \textbf{Assumptions:} $\beta_n - \beta_0 = O(n^{-\frac{1}{2}})$, $\hat{\beta}_n - \beta_0 = O_p(n^{-\frac{1}{2}})$; $ \hat{G}_{-i}(X_i'\beta_n) = G(X_i'\beta_n) + o_p(1)$; $\hat{G}_{-i}(X_i'\beta_0) = g(X_i'\beta_0) + o_p(1)$. \par}
\begin{enumerate}
	\item It can be shown $ S_{n}(\beta_n) = \frac{1}{n}\sum_i \{ g(X_i'\beta_0) - E[g(X_i'\beta_0)|X_i'\beta_n] +  \epsilon_i\}^2 + o_p(1)$.	
	\item With two Taylor expansions we have
$g(X_i'\beta_0) - E[g(X_i'\beta_0)|X_i'\beta_n)] 
 = g^{(1)}(X_i'\beta_n)( X_i - E[X_i'|X_i'\beta_n)(\beta_0 - \beta_n) + o_p(1)$
	\item Minimize $S_n$ in order to $\beta_n$.

\end{enumerate}

\end{frame}

%------------------------------------------------
\begin{frame}
\frametitle{Ichimura's (1993) method}
\begin{itemize}
		\item \textbf{Bandwidth Selection}:
Ichimura requires $h_n=O(n^{-\frac{1}{5}})$. H{\"a}rdle et al. (1993) suggest an empirical way of selecting the bandwidth for optimal smoothing of both $g$ and $\beta$.
		\item \textbf{Weight Function}:
	In case data is heteroskedastic, use analogue of Feasible Generalized Least Squares. Under certain regularity conditions, the efficiency bound for the single index model with unknown $g$ and using only data for which $X \in A_{\delta}$ is $\Omega_I$ with $w(x) = \frac{1}{\sigma^2(x)}$. The efficiency bound is then
		\begin{equation}
\Omega_{SI} = \left\{ E\left[\frac{1}{\sigma^2(x)}\frac{\partial}{\partial \beta}
 G(X'\beta,\beta)\frac{\partial}{\partial \beta} G(X'\beta,\beta) \right] \right\}^{-1}.
		\end{equation}
For unknown $\sigma^2(x)$, a consistent estimator is used that follows a two-step procedure.
		\item \textbf{Main Problem}: Uses iterative method, particularly difficult if the objective function is multimodal or nonconvex.
\end{itemize}
\end{frame}

%------------------------------------------------





\begin{frame}
\frametitle{Klein and Spady's (1993) method}

The model is defined as $Y_i =  \mathbf{1}{(X_i'\beta \geq \epsilon_i)}$. Let $g$ denote the distribution of $\epsilon_i$.
Thus, the $log-likelihood$ objective function is as follows:
\begin{equation}
\mathcal{L}_n(\beta) = \frac{1}{n}\sum_{i=1}^n \tau_{i}\{ (1 - Y_i)ln[ 1 - \hat{G}_{-i}(X_i'\beta)] +  Y_iln[\hat{G}_{-i}(X_i'\beta)]\}
\end{equation}
where $\tau_i$ is a simplified trimming function $\tau_i = \mathbf{1}{(X_i \in A_\delta)}$.
\begin{block}{Theorem Klein and Spady}
{\footnotesize \[\sqrt{n}(\hat{\beta}_{n} - \beta_0) \stackrel{d}{\rightarrow} N(0,\Omega_{KS}),
\] where $ \Omega_{KS} = \left\{ E\left[\frac{\partial}{\partial \beta}
 G(X_i'\beta)\frac{\partial}{\partial \beta} G(X_i'\beta)'\frac{1}{g(X_i'\beta_0)(1 - g(X_i'\beta_0))} \right]\right\}^{-1} $
 and $\Omega_{KS} = \Omega_{SI}$, i.e., the estimator is asymptotically efficient.\par}
\end{block}

\end{frame}


%------------------------------------------------
\begin{frame}
\frametitle{Klein and Spady's (1993) method}
\begin{itemize}
	\setlength\itemsep{1.5em}

	\item \textbf{Bandwidth Selection}: Klein and Spady require $ n^{-\frac{1}{6}} < h_n < n^{-\frac{1}{8}}$. H{\"a}rdle et al.'s (1993) solution can potentially be applied here.
	

	\item \textbf{Main Problem}: Computation is difficult.
	
	
	\item \textbf{Comparison between Ichimura's and Klein and Spady's model}:
	Klein and Spady's model seems more appropriate for the binary choice model case, from a theoretical perspective. Ichimura's model uses a weight function to correct for heteroskedasticity. However, Klein and Spady's model is efficient in the sense that it reaches the semiparametric efficiency bound.
	
\end{itemize}
\end{frame}

%------------------------------------------------

\begin{frame}
\frametitle{References}
\footnotesize{
\begin{thebibliography}{99} % Beamer does not support BibTeX so references must be inserted manually as below
\bibitem[Ichimura, 1993]{p1} Ichimura (1993)
\newblock Semiparametric least squares (SLS) and weighted SLS estimation of single-index models.
\newblock \emph{Journal of Econometrics} 58 71–120.
\end{thebibliography}
}
\footnotesize{
\begin{thebibliography}{99} % Beamer does not support BibTeX so references must be inserted manually as below
\bibitem[Horowitz, 2009]{p1} Horowitz (2009)
\newblock Semiparametric and nonparametric methods in econometrics.
\newblock \emph{Springer Series in Statistics} Springer.
\end{thebibliography}
}

\footnotesize{
\begin{thebibliography}{99} % Beamer does not support BibTeX so references must be inserted manually as below
\bibitem[Li, Q. and Racine, J. S, 2007]{p1} Li, Q. and Racine, J. S, 2007 (2007)
\newblock Nonparametric Econometrics: Theory and Practice.
\newblock \emph{Princeton University Press}.
\end{thebibliography}
}

\footnotesize{
\begin{thebibliography}{99} % Beamer does not support BibTeX so references must be inserted manually as below
\bibitem[Hardle, Hall and Ichimura, 1993]{p1} Hardle, Hall and Ichimura (1993)
\newblock Optimal Smoothing in Single Index Models
\newblock \emph{The Annals of Statistics}. 1, 175-178
\end{thebibliography}
}

\footnotesize{
\begin{thebibliography}{99} % Beamer does not support BibTeX so references must be inserted manually as below
\bibitem[Hardle, Hall and Ichimura, 1993]{p1} Hardle, Hall and Ichimura (1993)
\newblock Optimal Smoothing in Single Index Models
\newblock \emph{The Annals of Statistics}. 1, 175-178
\end{thebibliography}
}

\end{frame}

%------------------------------------------------

\begin{frame}


\footnotesize{
\begin{thebibliography}{99} % Beamer does not support BibTeX so references must be inserted manually as below
\bibitem[Klein and Spady, 1993]{p1} Klein and Spady (1993)
\newblock An Efficient Semiparameteric Estimator For Binary Response Models
\newblock \emph{The Annals of Statistics}. 2, 387-421
\end{thebibliography}
}
\end{frame}

%------------------------------------------------

\begin{frame}
\Huge{\centerline{The End}}
\end{frame}

%----------------------------------------------------------------------------------------

\end{document} 